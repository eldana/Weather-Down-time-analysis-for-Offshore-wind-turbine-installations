\begin{abstract}
%Contextualization
\textbf{Contextualization --}
The transport and installation of offshore wind turbines is highly dependent on weather condition at sea. any disturbance along the logistics chain could result in a significant delay in the project completion. 

%Gap - Altough ..., However ...
\textbf{Gap --}
However, a lot of research and development are still undergoing on how to develop a tool that could help install turbine components at higher weather restrictions (wind speed). 

%Purpose - This paper propose ...
\textbf{Purpose --}
The purpose of this paper is to carry out a weather down time analysis for offshore wind turbine transport and installations considering the weather restriction criteria for each activities along the logistics chain.  

%Methodology - .... results in ....
\textbf{Methodology --}
A discrete event  simulation model has been developed taking the vessel characteristics, distance matrix, installation methodology and sequence of activities into account in order to get the lead time completion. 

%Results - We suggest that ..., The results point to the development of ..., These findings provide ...
\textbf{Results --}
The results pointed out that the lifting operation causes higher down time over other activities and improving the lifting operations could result in a significant reduction in the overall project completion time. 

%Conclusions - We suggest that the
\textbf{Conclusions --}
a simulation weather down time analysis can improve the decision support system in the offshore wind energy development industry at the planning phase .
We suggest that simulations may improve and reduce the risks on the planning activities of offshore wind farms.

%\keywords{Discrete Event Simulation  \and Offshore \and Wind Farm \and Risk  \and Probability}
\end{abstract}
  