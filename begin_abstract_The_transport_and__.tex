
\begin{abstract}

The transport and installation of offshore wind turbines is highly dependent on weather condition at sea. Any disturbance along the logistics chain could result in a significant delay in the project completion. However, a lot of research and development are still undergoing on how to develop a mechanism that could help install turbine components at higher weather restrictions (wind speed). The purpose of this paper is to carry out a weather down time analysis for offshore wind turbine transport and installations considering the weather restriction criteria for each activities along the logistics chain. A Discrete Event  Simulation (DES) model has been developed taking the vessel characteristics, distance matrix, installation methodology and sequence of activities into account . The results pointed out that the lifting operation causes higher down time over other activities and improving the lifting operations could result in a significant reduction in the overall project completion time. This paper also  gives an insight how a simulation weather down time analysis could  improve the decision support system in the offshore wind energy development industry at the planning phase .

\keywords{Discrete Event Simulation , \and Offshore ,\and Wind Farm \and Risk , \and Probability}
\end{abstract}
  