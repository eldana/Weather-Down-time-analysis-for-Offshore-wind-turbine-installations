\begin{introduction}

The offshore wind energy has been progressing in the last 10 years. The annual offshore wind installation grows from 89.97 MW in 2004 to 1483.3 MW in 2014 (EWEA 2015). According to the European Wind Energy Association, 20 \% of the energy mix should come from renewable energy by 2020 (EWEA, 2007). Eventhough there exists a large amount of wind energy on European waters , it has several problems of huge cost of transport and installation facilities. It was discovered (Lange,Rinne & Haasis ,2012) that disturbances due to weather restrictions during the process of installing the turbine components at sea can lead to a significant increment of logistics costs. Green and Vasilakos (2011) noted that most of the costs associated with offshore wind energy development are still much higher compared to onshore counterparts. The need for utilizing expensive transportation resources gives rise to substantial difference between onshore and offshore operations. 

The transport and installation of offshore wind turbines is highly dependent on weather condition at sea. Any disturbance along the logistics chain could result in a significant delay in the project completion.Effective planning of the transport and installations procedures could help minimize the entire cost of the project completion.

\end{introduction}
