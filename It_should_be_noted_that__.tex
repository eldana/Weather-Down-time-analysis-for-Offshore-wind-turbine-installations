It should be noted that the start date of the project is considered to be the 20th of July for all the analysis presented in this paper and 400 simulation runs have been considered for each start date of the project in order to quantify the weather down time along the logistic chains. An activity time has been allocated for each operation where 15\%  is incorporated as a safety factor representing any delay due to personal inefficiency or machine breakdown and non weather related loss. If the time required to complete a certain operation is 4 hours, 4 + 4*0,15= 4.6 hours has been considered in the analysis. For the weather dependent activities, the time window required for computing the workability percentage is higher than the time required to complete the operation. This is very important  in risk reduction for major projects and temporary phase marine operations. some examples have been presented in the table 1 showing the activity time, weather window and weather restriction values. 

All the activities considered in this study have been divided in to four categories depending on the nature of the activity. 
\begin{enumerate}
\item
Lifting:- It refers to the activities like loading and installing the parts of the Offshore wind turbine.
\item 
Sailing:- It refers to the transportation phase of the project.
\item 
Jacking:- It refers to the activities related to positioning, jacking up and down.
\item
Cycle related activities:- It refers to the activities carried out per turbine at the offshore site before installing tower and after installing the last blade. 
\end{enumerate}


