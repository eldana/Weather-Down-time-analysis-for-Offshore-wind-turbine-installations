\subsection{Weather Condition}

The most important input parameter for the offshore transport and installation activities is weather data at a specific site. The weather time series data has been obtained from meteorological station for the period of 1995-2008 for wind speed and wave height. Both historical weather data or probabilistic approach may be used so as to  analyze the project lead time of completing the installation of the Offshore wind turbines. Muhabie et al (xxxx) compared the installation offshore wind farms based on discrete event simulation using both historical weather time series and probabilistic approach  and it was found out that both approach shows a good agreement. In this study a probabilistic approach has been implemented where the monthly probability of working and non working is computed based on the weather restriction and time window criteria for a specific activity. 
\begin{enumerate}
\item
Workability:- it refers to the condition above which an operation can not be carried out any more (could be wind speed or wave height or both).
\item Time window:- It refers to the good weather condition in order to complete an operation and is simply the range of workability.
\end{enumerate}
The weather condition at see changes randomly from time to time and different scenarios have to be taken in to account. The most common practice in the offshore industry is to have the following considerations.

\begin{enumerate}[label=\roman*]
\item
Best weather condition(Q20): Under this assumption the weather is assumed to be at its best condition in order to perform an operation and can be computed using the complementary cumulative probability function. 
\item
Average weather condition (50): this simply refers to the mean value of the monthly working percentage and it also represents to the mean of the normal distribution.
\item Worst weather condition (Q80):- Under this assumption the weather is assumed to be at its worst condition in order to perform an operation and can be computed using the complementary cumulative probability function. 
\end{enumerate}

