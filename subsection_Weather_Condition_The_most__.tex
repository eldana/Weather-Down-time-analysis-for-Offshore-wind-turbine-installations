\subsection{Weather Condition}

The most important input parameter for the offshore transport and installation activities is weather data at a specific site. The weather time series data has been obtained from meteorological station for the period of 1995-2008 for wind speed and wave height. Both historical weather data or probabilistic approach may be used so as to  analyze the project lead time of completing the installation of the Offshore wind turbines.Weather predictions and numerical weather forecasts can be calculated with different models. However, the reliable weather predictions are mostly provided for a period of approximately a maximum of 14 days, \cite{hinnenthal2007}. This is obviously not appropriated for a long-term scheduling. Muhabie et al \cite{Muhabie2015} compared the installation offshore wind farms based on discrete event simulation using both historical weather time series and probabilistic approach  and it was found out that both approach shows a good agreement. In this study a probabilistic approach has been implemented where the monthly probability of working and non working is computed based on the weather restriction and time window criteria for a specific activity. 
\begin{enumerate}
\item
Workability:- it refers to the condition above which an operation can not be carried out any more (could be wind speed or wave height or both).
\item Time window:- It refers to the good weather condition in order to complete an operation and is simply the range of workability.
\end{enumerate}
It is known that the wind measurement is taken at a specific height and there should be a way to find the wind speed at a working condition at the offshore site. 
The wind profile power law relationship, presented in equation \ref{eqn:windprofile}, is used to estimate the wind speed $u$ at height $z$, when $u_{r}$ is the known wind speed at a reference height $z_{r}$, \cite{1978Peterson}. The exponent $\alpha$ is an empirically derived coefficient that varies dependent on the stability of the atmosphere.The shear exponent $\alpha$ varies depending on atmospheric conditions, temperature, pressure, humidity, time of the day and nature of terrain\cite{Manwell2009}. The shear component can typically be assumed to be equal to 0.1 in offshore environment (Bechrakis and Sparis, 2000; Burton et al., 2011).

\begin{equation}
\label{eqn:windprofile}
u = u_{r} \left( \frac{z}{z_r} \right)^{\alpha}
\end{equation}

it is assumed that the percentage of monthly workability assumes normal distribution over time. In order to  generate the normal distribution,the mean and the standard deviation of the percentage workability computed, have been taken in to account. The weather condition at see changes randomly from time to time and different scenarios have to be taken in to account . In this paper three different weather scenarios (Best, Average and worst) have been considered and computed applying the cumulative distribution function (CDF) and its inverse distribution function (quantile function).The cumulative distribution function of the workability (for each month) can be expressed by \ref{eqn:cfd}

\begin{equation}
\label{eqn:cfd}
F\left( x \right) =P\left(X<=\right),

\end{equation}


 where the right-hand side represents the probability that the random variable X takes on a value less than or equal to x. The quantile function (for example Q80) can be computed by finding the value of x such that F(x)=0.2 .  

\begin{enumerate}[label=\roman*]
\item
Best weather condition(Q20): Under this assumption the weather is assumed to be at its best condition in order to perform an operation and can be computed using the complementary cumulative probability function. 
\item
Average weather condition (50): this simply refers to the mean value of the monthly working percentage and it also represents to the mean of the normal distribution.
\item Worst weather condition (Q80):- Under this assumption the weather is assumed to be at its worst condition in order to perform an operation and can be computed using the complementary cumulative probability function. 
\end{enumerate}

